\textbf{X}:
\href{http://www.lescienze.it/news/2015/05/16/news/discussione_fisici_filosofi_filosofia_morta_viva-2611425/}{link}.

\textbf{tetrapharmakon}: parte della risposta che ti darei di persona
l'ho scritta lì: la filosofia ha palesemente fallito, perché non ha
notato che il mondo è terribile e simmetrico, non ci vuole ma esige
esegeti (perdona la sciocca figura di suono). In questa dicotomia
(bibliotecari coatti di un Universo che ci respinge) rimaniamo
stritolati. Qualcuno per fortuna a volte si solleva e con l'arte, le
lettere o il maneggio di una katana produce piccole gemme. ~ Questa
metafisica la trovo più elegante del nichilismo (perché non annulla il
mondo), ma allo stesso tempo meno mielosa del pensiero riprovevole che
esista un'entelechia. L'ha sposata Borges, l'hanno presentita molte
persone perdendo la vista. Probabilmente il cervello dice cose che noi
non possiamo capire, a un cieco. In ciò la matematica per me è stata una
salvezza, perché dedicarmi al sacerdozio di un culto che si ostina a
rifiutarmi mi ha insegnato cose sui moventi dell'animo, molto più della
filosofia o dell'etica. Oggi sono verboso.

\textbf{X}: Verboso è dir poco. Mah. La filosofia ha fallito. Mah. Mah.
Mah.

\textbf{tetrapharmakon}: ha \emph{palesemente} fallito, mi dispiace

\textbf{X}: La filosofia delle scienze forse.

\textbf{tetrapharmakon}: stai davvero cercando di convincermi che
esistano un'etica e una morale?

\textbf{tetrapharmakon}: L'estetica è una categoria di giudizio
scientifica, la metafisica un ramo della letteratura fantastica, etica e
morale teratomi del Super-Io, tutto ciò che ha anche la minima
intersezione con qualsiasi materia scientifizzabile (dove, cioè, il
linguaggio matematico offra modelli predittivi) è fatta meglio dagli
scienziati.

\textbf{tetrapharmakon}: forse dimentico qualcosa, ma. ~ Se vuoi faccio
un emendamento comunque: magari la filosofia non ha fallito, ma scopa
proprio male

~ Sophia è una a cui piace il fisting, chi l'ha capito l'ha fatta venire
come una fontana, gli altri hanno optato per un timido onanismo.

\textbf{tetrapharmakon}: Comunque non mi aspettavo, da te, questa vis
conservatrice mi sorprende

\textbf{X}: Ma perché riconduci tutto a uno. Un'etica e una morale,
l'etica e la morale. Sull'estetica non sono per nulla d'accordo, e lo
sai. La metafisica così' detta è proprio da scienziati, mentre io la
considero preziosa fonte di indagine. Il punto è che, nella mia
piccolissima esperienza, non ho mai sentito il bisogno di porre scienza
e filosofia sullo stesso piano, e se mai dovessi farlo, considererei
tutto filosofia, e le discipline suoi rami. Tuttavia ritengo che fare
filosofia sia saper pensare e porsi buone domande, e di conseguenza
primariamente indagine su etica e morale. Il valore arbitrario non è poi
così arbitrario nel linguaggio, ma il linguaggio è pensiero e il
pensiero è filosofia. Sia chiaro, questo vale per me.

\textbf{X}: non sono né una filosofa né una matematica né una storica

\textbf{tetrapharmakon}: Per lo più citavo Borges rispondendo (Tlon,
Uqbar, Orbis Tertius). Ora però ti rispondo seriamente, o almeno ci
provo. ~ Si può parlare del ``fallimento'' della filosofia, o della
parte di filosofia che viene insegnata e studiata qui in Italia, nel
momento in cui la poca lungimiranza gentiliana, di quell'abominevole
cancro del pensiero occidentale che sono stati i seguaci di Hegel, hanno
cominciato a promanare le loro ideologie malsane che davano alla scienza
il ruolo di ancella minore della Storia, Unica Maestra. ``Gli ingegni
minuti'', li chiamava Vico gli scienziati, e quel vomitevole aborto
acefalo di Croce gli ha fatto eco: ``qui incipit numerare, incipit
errare!'' ~ Federigo Enriques, che ebbe un carteggio molto acceso con
Gentile, era invece di quella scuola per cui la matematica, ``infezione
benigna del pensiero filosofico'', è il complemento naturale degli studi
umanistici di un individuo. Il complemento: non il palliativo, non la
sostituzione, ma l'avente pari diritto ontologico. ~ Gentile ha vinto,
riformando la scuola ed edificandola su un putrido neoidealismo; il
prezzo lo pagano tutti gli studenti o anche gli adulti che oggi mi
dicono ``io la matematica non la capisco: non l'ho mai amata tanto, non
vedo come una soluzione sia nel suo assoluto corretta e l'unica da dover
raggiungere. ~ Insomma, l'uomo ha cosí tante sfacciettature, come puó
indugiare su una sola?'' ~ A queste persone io dico: ~ Esiste una
dimensione creativa, mistica, nella scienza. Perché esiste una
dimensione artistica, direi sessuale, nella matematica, che i perbenisti
non lasciano trapassare nei manuali liceali ``ad usum delphini'',
epurati da qualsiasi traccia di arte, sangue e frammenti di cervella
esplose da un colpo di revolver. ~ La matematica per me è l'isola serena
in cui la smania del mondo col suo sudore e le sue rivolte, i fumi del
suo alcool e le strida scimmiesche smettono di toccarmi; è la strada che
mi permette di leggere criticamente la Qabbalah, i testi gnostici, il
Tripitaka, Sepúlveda, Einstein e il Necronomicon allo stesso tempo,
dandomi la possibilità di parlare dell'unica convinzione del me
adolescente che non mi ha mai abbandonato: il fatto che tutte le
discipline umane (quindi non solo la scienza, e le varie parti della
matematica) siano unite da un unico filo rosso: lettere, arti, politica,
scienza, religione, filosofia. ~ E io non sono un un caso a parte: ho
amici che leggono Heidegger, i testi gnostici e la Torah ``perché gli
va'', che suonano il piano, che amano l'arte e le lettere, che
partecipano (e intervengono pure!) agli After Talks, e che magari ci
provano con l'organizzatrice dopo il dibattito. ~ Viceversa, ogni volta
che ho conosciuto un filosofo, ho fatto fatica a fargli capire persino
che esistono infiniti numeri primi; è una competenza che non posso
aspettarmi da lui, anzi spesso ricusa persino la possibilita' di farsi
una cultura scientifica, ``che tanto è roba che non capisco'' e ``io
sono fiero di non capirla la matematica''. ~ Rifletti: tu sai forse il
latino, e sai di cosa parlano i promessi sposi, ma NON SAI che esistono
infiniti numeri primi. E' vergognoso e castrante. Ti è vietata la
speculazione sull'origine dei numeri; ti è preclusa la comprensione
della geometria, e del fatto che intimamente, fare geometria e fare
logica (ovvero un ramo della filosofia, almeno a giudicare dalla
tassonomia usuale) sono la stessa cosa. Non cose simili. Non analoghe.
LA STESSA FOTTUTA COSA, lo stesso campo da gioco. ~ Sono fighissimi gli
artisti che si fanno murare in un loft per qualche giorno per celebrare
il proprio distacco dalla societa' perbenista viennese. D'altra parte
questo tipo di eccessi li avevamo anche noi: ~ Georg Cantor è morto in
un manicomio dell'alta Sassonia, dopo essere stato scomunicato per aver
chiamato ``Ω'' le quantità infinite di cui aveva ideato le regole
aritmetiche (sì, si può sommare l'infinito a un infinito più grande, che
lo fagocita; e si può sottrarre un infinito a un infinito più grande,
che non lo scalfisce di un millimetro). ~ Archimede fu passato a fil di
spada da un soldato romano, che calciò la sabbia dove il greco stava
disegnando figure geometriche : Archimede gridò al soldato ``Noli
tangere circulos meos!'' senza nemmeno guardarlo in faccia. ~ Pitagora
ordinò l'esecuzione per annegamento di un discepolo della sua setta che
voleva rivelare al mondo che la diagonale di un quadrato di raggio 1 è
un numero che non si può scrivere come rapporto di interi. Questa
scoperta ripugnava alla mentalità greca per cui ogni cosa è soggetta ad
una proporzione che la descrive: il mondo è un numero. ~ Évariste Galois
morì in un duello alle pistole, generato da una lite per l'amore di una
prostituta di Pigalle, a 22 anni, dopo aver passato la sua ultima notte
a redigere febbrilmente la summa del suo lavoro (iniziato quando aveva
19 anni, il più giovane studente del politecnico di Parigi), ubriaco e
stordito dal sonno e dalla paura di morire. ~ Paul Erdos visse più di
ottant'anni, facendo la vita del clochard e facendo costante uso di
benzedrina e pastiglie di caffeina, senza nessun avere terreno (i suoi
pochi indumenti erano stipati in una valigia lisa che portava ovunque).
Si autoinvitava a casa dei colleghi, spesso senza alcun preavviso,
gridando ``la mia mente è aperta, lavoriamo!''. ~ Quando il figlio
primogenito di David Hilbert venne portato in manicomio sotto i suoi
occhi, Hilbert continuò la lezione che stava tenendo ai suoi allievi; e
quando gli comunicarono che uno studente molto brillante aveva
abbandonato la matematica per fare lo scrittore Hilbert proruppe in una
frase molto famosa: ``Molto meglio che faccia lo scrittore: non aveva
abbastanza fantasia per fare il matematico''. ~ Einstein e Russell erano
dichiaratamente poligami, o comunque hanno avuto per lungo tempo
relazioni aperte quanto e come le flapper o le poete surrealiste.
Parliamo poi del suicidio di Turing, oppure del fatto che Ted Kaczynsky
fosse un matematico (anche piuttosto promettente). ~ Quando a Grigorij
Perel'man venne offerto un milione di dollari per la soluzione di uno
dei ``problemi del millennio'', alcuni tra i problemi di matematica più
difficili ancora insoluti, lui rifiutò e disse che la commissione che
voleva premiarlo era ``incapace di giudicare il suo lavoro'' e che ``il
denaro, oggi, porta solo guerra e invidia''. Vive in una casa popolare
di Mosca, da recluso. ~ Malattia, follia, sesso, amore, morte, droga e
stati mentali alterati sono la faccia umana della Matematica.
Indissolubilmente legata all'arte, alle lettere, alla psichiatria, al
dolore fisico e mentale, alla parte più atra e ventrale del nostro
spirito. ~ \ldots ora sai che se l'anno prossimo parlo io, non ce n'è
uno che va a casa intero.
